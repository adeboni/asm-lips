\chapter{Audience Analysis}

Our audience consisted of three groups. The first was technical and very knowledgeable of our research area. The second were those who were there to judge us on our writing and oral presentations. The final group was comprised of those with little to no technical knowledge, including the people who might use the final application that our library runs in as well as our family and friends that attended to see us present.

\section{Judges}
The judges of our presentation were the most important members of our audience, as they were, after all, the ones who inevitably decide how well we do. They probably had a basic understanding of general engineering concepts, but most likely not any knowledge specific to our project's field. As a result, we had to explain the concepts to them in a way that they could understand. A combination of concise text and graphics was useful in this case.


\section{Technical}
The main member of this group was our project advisor. She had the strongest technical understanding of the material that we presented. Others with her experience and understanding of the subject matter were in the audience as well, such as our advisor's colleagues or other professors. Members of this audience group likely understood the subject matter of our presentation as well as we do if not better.


\section{Users and Family/Friends}
The people using the language learning application that our library powers want it to always work reliably. If it doesn't they will be upset and complain. As for family and friends, they wanted to support us and try to learn about what we made. Both parties had very little to no knowledge of how our system works, so explanations to them needed to be very simple and straightforward, otherwise we risked boring or confusing them. Pictures, rather than text were incredibly beneficial here.

\section{Conclusion}
Much of our presentation was be targeted toward the judges, since they have the most impact on our success. We started out general by giving a simple, interface and functionality-based overview of the whole system. This allowed the people who didn't want a lot of complicated information to at least understand what our project is about. Once the general information was covered, we went into more information for the judges and other, more technical, audience members. Given that the judges were the primary focus of our presentation, we did not spend too much time on the most heavily technical sections so that we did not lose their attention in extensive computation and field specific jargon.